\documentclass[10pt]{article}
\usepackage[margin=1.1in]{geometry}
%\usepackage[small,compact]{titlesec} %very powerful
\usepackage{tcolorbox}
\usepackage{enumitem}
\usepackage[T1]{fontenc}
\usepackage{cite}
%\usepackage{caption}
%\captionsetup{font=small}
\usepackage{graphicx}
\usepackage{wrapfig}
\usepackage{amsmath}
\usepackage{amssymb}
\usepackage{amsthm}
\usepackage{hyperref}
\usepackage{xcolor}
\usepackage{hyperref}
\hypersetup{
    colorlinks,
    citecolor=black,
    filecolor=black,
    linkcolor=blue,
    urlcolor=blue,
}
\usepackage{booktabs}


\def\Section{\S}

\newcommand{\mycomment}[3][\color{black}]{{#1{{#2}: {#3}}}}
\newcommand{\tvn}[1]{\mycomment{TVN}{#1}}{}
\newcommand{\red}[1]{{\color{red}{#1}}}
\renewcommand{\figurename}{Fig.}
\renewcommand{\tablename}{Tab.}
\title{Demystifying the Computer Science Ph.D. Admission in the US}
\date{}
\author{ThanhVu (Vu) Nguyen, George Mason University}
%\setcounter{secnumdepth}{3}
\begin{document}
\maketitle

\begin{abstract}
Having been involved in PhD admissions for many years and after
numerous interaction with  students (especially Vietnamese), I've
realized that \emph{international} students lack a clear understanding of
the Computer Science PhD admission process at US universities. This confusion not only
discourages students from applying but also creates the perception that
getting admitted is difficult compared to CS PhD programs in other countries.

So I want to share my opinions and advice for those who are interested in applying for a \textbf{PhD in Computer Science in the US}.
While I wrote this document for CS applicants, it should be applicable to various STEM fields.
Moreover, while many examples given are for GMU, which is 33rd on CSRankings (June 2016), the writing should be generalize to most other R1\footnote{An \href{https://en.wikipedia.org/wiki/List_of_research_universities_in_the_United_States}{R1 institution} in the US refers to a research-intensive university with a high level of research activity across various disciplines.} universities  (though \emph{very} top schools might be very extreme, e.g., see the admission process at \href{https://da-data.blogspot.com/2015/03/reflecting-on-cs-graduate-admissions.html}{CMU}).

I wish you the best of luck. And if you follow these advice,
you will at least have a good chance at GMU (see
\href{https://github.com/dynaroars/dynaroars.github.io/wiki/About-GMU}{why
you want to study at GMU}). Happy hunting!

If you have suggestions or comments, feel free to create a \href{https://github.com/nguyenthanhvuh/phd-cs-us/issues}{GitHub issue} for discussion.
\end{abstract}

\section{Should You Apply?}


First, I want to emphasize that PhD students in Computer
Science \emph{do not} need to worry about funding, especially at R1
universities in the US. If you are admitted, you will almost certainly
receive \emph{full funding} to support your study, including tuition,
health insurance, and 9-month stipend. Moreover, depending on the university,
you may even receive additional benefits like summer salary. Note that
funding is typically more available for PhD students than 
Masters. % [[#funding][Read more about funding here]]

Second, I believe that applying to a good US university \emph{should not} be any
harder than at good schools in other countries. If you think you have a
chance at schools in other countries, e.g., South Korea, Singapore, Germany, UK, Japan and Australia, then you surely have a chance in
the US as well.

\begin{tcolorbox}[left=1pt,right=1pt,top=1pt,bottom=1pt]
Vu: One of the reasons I create this post is that several of my colleagues are interested in 
recruiting Vietnamese students and were surprised when seeing very few Vietnamese applications (compared to other countries). In general the number of
PhD applications from Vietnam to US universities is very low and  more would be very welcomed. 
\end{tcolorbox}

%% Vu: *what's a PhD?*  This [[https://matt.might.net/articles/phd-school-in-pictures/][series of pictures]] from [[https://matt.might.net][Matt Might]] illustrates what a PhD means.
%% \end{tcolorbox}


\section{How is Your Application Evaluated?}

After you submit your PhD application (usually in December), it will be first screened
for general requirements, e.g., did you submit your transcripts and standard scores? did your reference writers submit their letters?
Then your application will be reviewed by a
\textbf{PhD admission committee} consisting of faculty members in CS. Each application is assigned to about \emph{three} faculty members, who will evaluate your profile and try to reach a consensus about your case.  Note that while in most cases the assigned reviewers will be the main ones deciding your application, every faculty will have access to your application and can give inputs on your profile.

In many cases, the admission committee will involve junior faculty (e.g., assistant professors) in the department. While this requires  a significant time investment, it provides junior faculty the opportunities to recruit students. The chair of the committee will be a senior professor, but they will not review individual applications and instead assign them to committee members. The chair will look at various factors such as research interests or mentioning faculty names, and assign the applications to appropriate faculty. 

\begin{tcolorbox}[left=1pt,right=1pt,top=1pt,bottom=1pt]
Vu: We usually decide that a candidate is either (i) admit with funding (TA or RA) or (ii) rejected. In other words, in most cases, we either
admit you with full funding, or we don't. In some rare cases, we may admit
you with no funding because you have funding on your own (e.g., you are
supported by your government or have external grants). Also, we justify
our decision with a short summary about your application, where we list
strengths (e.g., well-known undergrad school) and weaknesses (e.g., weak
LORs).
\end{tcolorbox}

\begin{tcolorbox}[left=1pt,right=1pt,top=1pt,bottom=1pt]
    Hakan: At GMU, for full consideration, students should make sure to submit \textbf{ALL} required documents by the application deadline (December 1st), and should never assume that some required documents (such as official TOEFL scores or official diplomas/transcripts) will be waived by the admissions office. If something is listed and not marked as "optional", it is mandatory and they should plan for submitting all those.  
\end{tcolorbox}
\paragraph{Why we do not waive application fee?}  This is typically a requirement of the university. Individual departments/programs do not have the flexibility to waive the application fee, even if they want to. 

In my opinion, requiring applicants to pay the fee helps ensure their seriousness, as it filters out non-serious candidates. Also, if the application process were free for everyone, we would receive an overwhelming number of applications to review.


\section{Application}\label{sec:application}

The primary focus of the admissions committee is to \textbf{evaluate your background and interest in research} since a PhD in Computer Science
is a research degree. To assess your research capability, we consider
the following key indicators, listed in order of importance.

\subsection{Research Ability}

The most effective evidence of research ability is having \textbf{published papers in reputable international conferences or journals}.
Having published good papers is a sign that the applicant was involved in research.

It's important to aim for the top conferences in your field, which you can
find on places such as CSRankings~\cite{csrankings}, designed specifically to help CS PhD
applicants. Local conferences and non-English journals or conferences do
not carry as much weight since their quality is often unknown to US
faculty.

However, I understand that many international students do not have the opportunities to publish in top places, so general IEEE/ACM/USENIX confs/journals would suffice.  But be sure to talk about it in your \hyperref[sec:research-statement]{statement}.

\begin{tcolorbox}[left=1pt,right=1pt,top=1pt,bottom=1pt]
Vu: Vietnamese students often mention Scopus Q1, which consists of diverse journals from IEEE, Elsevier, and several other publishers unfamiliar to me.  Honestly, I don't know/recognize the majority of journals listed in Scopus Q1. So this might be something to be mindful of, as \textbf{CS} faculty might not be too familiar with Scopus or journals listed in there, so devote sometime in your statement to discuss the significance of your papers.
\end{tcolorbox}

\begin{tcolorbox}[left=1pt,right=1pt,top=1pt,bottom=1pt]
Craig: GMU and many other universities allow you to upload your published papers and other writing samples. In many cases, even if the papers were not published at top places, we can still determine their quality by simply skimming over the paper.  
\end{tcolorbox}

Additionally, \textbf{work experiences at renowned research laboratories}, such as Microsoft Research, can significantly strengthen your
application.  Unfortunately, many good international research places, e.g., VinAI in Vietnam, remain relatively unknown to most universities in the US. So you should explicit say something about them in your statement.

Finally, \textbf{participating internationally recognized competitions} can also demonstrate your research potential.
For example, participating in Math Olympiads if you want to do theory or  winning ACM programming contests if you want to ``build'' stuff, e.g., software analysis.


\subsection{Letters of Recommendation (LORs)}

Most CS PhD applications will require at least \textbf{two LORs}. Having a letter from an internationally recognized researcher can greatly strengthen your application. However, obtaining such letters
can be challenging for international students, who might not have much interactions with such experts. So it is acceptable to have a letter from professors that know you well enough to talk about \emph{your specific research experience and capabilities}.


Many students have letters written by the applicants themselves and signed by their professors. These have little
values (we can easily recognize them) and will consider them weakness.
Similarly, many professors write very generic letters for students (a common example is that the students didn't do any
research or make any impression for the professor to write about). These
letters are also not useful and considered weak.

Many students get letters from someone from company where they did internship or are
working at. It is OK as long as it is a research-based personalized
letter (once again, we are talking about PhD applications, not MS).


\begin{tcolorbox}[left=1pt,right=1pt,top=1pt,bottom=1pt]
Vu: For me personally, it's better to have a good personalized
letter about your own research ability from someone who is less
well-known than a generic/weak letter from a well-known person.
\end{tcolorbox}

\subsection{Research Statement}\label{sec:research-statement}

While you might not be able to get good LORs or change
\hyperref[sec:your-school]{where your go to school}, you have control of your
statement! So write it well because we do take it seriously.
A well-written LOR also shows that you can communicate, which is very important in research, and that you can effectively teach and communicate with students, which is important for TA funding (see \S\ref{sec:funding}).
%is important because if you need (GTA) funding, it will provide evidence
%that you can teach and communicate with students.

There are many guides on writing research statement, e.g.,~\cite{blattman2022writing},
so I will not talk too much about it. In short, discuss about your research vision and convince us that you can achieve it through your experience, e.g., published papers, or if you work on some projects by yourself, talk about it.

Finally, this is something very easy to do, but is missed by many
applicants: \textbf{customize the statement} for the school you're applying to,
e.g., why do you apply here? talk about a couple of professors who you're interested in working with (in many cases your application will be forwarded to them for evaluation).
Be careful not to send wrong statement to wrong school or mixing
facts (e.g., talking about school X but mentioned about working with
profs. at school Y; and definitely do not talk about George Washington when applying to George Mason). I have seen such statements more time that I
should.


\begin{tcolorbox}[left=1pt,right=1pt,top=1pt,bottom=1pt]
Vu: I always read the research statement first and then LORs. If I am
persuaded by then, I would skim over other factors and advocate for
admission (unless I see red flags in other parts). If I am not
convinced, then I will likely recommend rejection (unless I see
something standout in other parts).
\end{tcolorbox}


\subsection{Your School}\label{sec:your-school}

Graduating from top universities \emph{that we recognize} helps.
However, if committee members do not know much about schools in your country, they will likely treat your school as
\emph{``unknown foreign''}, which can be a minus point (if your school is well-known, then it is \emph{``top foreign''}, which is definitely a plus).

So what can you do about this? several things such as asking your CS dept to put itself on CSRankings (it's the easiest way to get CS people to know about the school), explaining about your school in your statement (and asking your LOR writer to do that too), and of course, if you're Vietnamese, considering a CS PhD program that has \href{https://github.com/dynaroars/dynaroars.github.io/wiki/Viet-CS-Profs-US}{Vietnamese professors}.

\begin{tcolorbox}[left=1pt,right=1pt,top=1pt,bottom=1pt]
Vu: Sometime PhD admission committee will share a document such as \href{https://github.com/dynaroars/dynaroars.github.io/wiki/Foreign-Top-Schools}{this one}, which lists the top schools in several countries. I have looked at Vietnamese applications (whether they are assigned to me or not) and provide inputs to the reviewers of those applications, e.g., X is the top tech school in Vietnam and so it should be \emph{top} instead of \emph{unknown} foreign, which makes a huge difference.
\end{tcolorbox}

\subsection{Grades/GREs}\label{sec:grades}
Having good grades is important, but, in general, unless your school is well-known, having top grades/ranks
usually will not help. This is simply because we cannot evaluate them.

This can be an issue for students in many top international universities where the competition is so high that very good students students can still have low ranking (and be overlooked by Admission committee).
So what to do with this? well, same as \hyperref[sec:your-school]{before}, e.g., put a note about this in your statement.

Note that while having good grades at unknown school might not help,
having very bad grades will be \red{red flag} (unless your LORs or
statements give proper explanation). This is especially true if you
have bad grades in relevant courses.

\paragraph{GRE} Most CS programs in the US \emph{no longer require GREs}, so you don't need to
take them. However, they might be useful for international students from programs we are not familiar with. 

You also do need to
take standardized English test. Just do well enough to pass minimum
requirement set by the university.

\begin{tcolorbox}[left=1pt,right=1pt,top=1pt,bottom=1pt]
Vu: The minimum for GMU CS (being above this might not mean much, but below is a \red{red flag}).
\begin{itemize}
\item GPA: $\ge 3.0$ in your undergrad (but as mentioned, we also consider the rank/prestige of your school)
\item GRE: not required (most CS graduate programs no longer require GRE, though it can help boost your profile)
%    \item but if you want to use it, then we expect a total (V+Q) of $\ge 311$ (with a $\ge 157$ Q) and A $\ge 3.0-3.5$.
\item English requirement tests (one of the below)
  \begin{itemize}    
  \item TOEF: 88 pts in total AND $\ge 20$ points in each subsection OR
  \item IELTS: $\ge 6.5$ OR
  \item DuoLingo Graduate English: $\ge 120$ OR 
  \item Pearson Test of Academic English: $\ge 67$
  \end{itemize}  
\end{itemize}
\end{tcolorbox}


\subsection{CV/Resume}
This should be a summary of the accomplishments of the applicant.  I would quickly scan this document to see if something stand out (e.g., Publications, Programming Competition Awards, Teaching Experiences).

\subsection{Interview}

Sometime a faculty wants to interview an applicant to make a decision. Typically this means they lean toward admitting you (if we don't like your application, we will not bother doing the interview).

An interview lasts about 15--30 mins, and one implicit thing you will be evaluated on is whether you can communicate effectively (i.e., speak/understand English).  You will also get chance to ask questions about the university so think of something to ask (just the same as you interview at a company).

\begin{tcolorbox}[left=1pt,right=1pt,top=1pt,bottom=1pt]
At GMU, we are encouraged to interview candidates. For very strong candidates, the interview is actually to recruit them.  In some rare cases a faculty interviews a candidate that they see potentials and want argue for admission, i.e., without the interview, that candidate is definitely rejected. In any case, getting interview means you have a very good chance of being admitted.
\end{tcolorbox}

\section{Getting Admitted and Choosing the Right School}

Around March you should hear back from most PhD programs that you applied (if not, send email and ask). You will have to make your decision by around April 15.
If you have offers, congratulations!  Now you're at a different game because the schools that admit you will now try to get you to accept them!  

Most schools will have an \textbf{Open House}, which is a great resource to learn about the school, department, faculty, research, living, etc. During the Open House, you get a chance to talk to individual faculty and current students.  Take notes of faculty who make you excited, count those that are taking in new students (if they meet you, likely they are considering new students!).  Talk to students about their advisors, the dept, the area, funding situation etc.  Ask about anything you want to determine that they deserve \emph{you}.

In short, if you can come to the Open House, do come.  But if you're international student outside of the US, then likely you cannot come.  So see if you can attend it virtually and ask to meet with individual faculty if you can.

\begin{tcolorbox}[left=1pt,right=1pt,top=1pt,bottom=1pt]
Vu: GMU has Virtual Open House, e.g., \url{https://cs-gmu.github.io/cs-phd-voh-s23/}, which I've co-organized in the last two years. We invite all admitted PhD students to the VOH through Zoom to learn about the CS program, the department, GMU, and the DC area in general. Students also get opportunities to chat with professors and current students.
\end{tcolorbox}



\section{Funding}\label{sec:funding}

As mentioned, if you're admitted to a \emph{good} CS PhD program, you should not have to worry about funding!  
In the US, the common types of funding for PhD are \emph{graduate teaching assistant} (GTA or TA), \emph{graduate research assistant} (GRA or RA), and \emph{Fellowship}.
RA is the type of funding paid by a prof. for you to do their research. TA is paid by the department for you to help with teaching. Finally, fellowship is an independent funding that can come the school, a company, or organization. Tab.~\ref{tab:funding} summarizes the differences. 

\begin{table}
  \centering
  \caption{Different types of PhD funding}\label{tab:funding}
  \begin{tabular}{c|c|c|c}
    \toprule
    &\textbf{TA}&\textbf{RA}&\textbf{Fellowship}\\
    \midrule
    \textbf{From} & School & Profs. & School/External\\
    \textbf{For}                  & Teaching Assistant       & Research                        & Research                              \\
    \textbf{Tuition/Ins./Stipend} & Yes                      & Yes                             & Yes                                   \\
    \textbf{Cover Summer?}              & No                       & Maybe                           & Yes                                   \\
    \midrule
    \textbf{Pros}                 & Research Freedom         & Get to do research              & Research Freedom                      \\
    \textbf{Cons}                 & TA, Uncertain            & Research restriction, Uncertain & Competitive, limited             \\
    \bottomrule
  \end{tabular}
\end{table}

\subsection{Graduating Assistantship (TA/RA)}
The most common type of funding is \textbf{graduate assistanship}, which is either TA or RA. Both TA and RA come with tuition waiving (you don't have to pay tuition), health insurance (this takes care of your insurance, which is a must have in the US), and most importantly, your stipend (i.e., your salary). Some universities also pay insurance for spouse/children (or give very good discount).

Several about stipend. First, the amount of stipend depends on the university, which in turns depend on various factors such as location (e.g., a stipend in Washington DC is likely higher than in Lincoln, Nebraska due to higher living cost). Second, a school year is (typically) 9-month in the US, so stipend is for 9 months (so divide by 9 for each month). Third, like for most source of income in the US, you will have to pay tax on your stipend. Finally, private universities might pay more for stipend.

\begin{tcolorbox}[left=1pt,right=1pt,top=1pt,bottom=1pt]
At GMU, TA and RA have similar benefits in tuition waiving and insurance.  For stipend, depending on the college and department, a 9-month graduate assistant stipend is set.  TA and RA will usually be that amount (TA will definitely be that, RA might fluctuate a bit depending on the stage of the student and the prof.). 
\end{tcolorbox}

\subsubsection{Teaching Assistant (TA)}

TA is common in the beginning when you haven't found your advisor who would pay you RA. As a TA, you help professors with their classes (e.g., grading or teaching labs/recitation). Your TAship is paid through the department, i.e., they hire you to help teach.  During a semester, a TA might work with several courses and professors (not necessary their advisor).  TA funding typically is not available during the summer, which has no school.

\paragraph{How to get TA?}  Unless you have other funding such as RA or Fellowships, TA is typically a default thing. When you apply to be a full-time student, you also sate that you need financial assistant. It is common that the PhD committee will either admit you and give you GTA, or reject you; i.e., we do not admit a student without supporting them.  

\begin{tcolorbox}[left=1pt,right=1pt,top=1pt,bottom=1pt]
At GMU CS, students admitted with TA have  4 years of GTA guaranteed and also receive  stipend for the \textbf{first} summer.
\end{tcolorbox}

Even if you have other funding and do not need TA, you still should do TA at least once.  This allows you to see what teaching is like, which is especially helpful for research career where you often have to give talks and tell people about your work. Note that GMU sometimes has classes that a more senior student can teach.  In that case, you will be paid as a lecturer, which is higher than GTA.  This is a good opportunity for students to get teaching experience and also get paid more.
\subsection{Research Assistant (RA)}
RA is provided through a professor through their own funding so you can work on their project.  
You do not need to teach as an RA, so you can focus on your research. Depending on the professor, RA may be available during the summer.

\textbf{How to get RA?} When a professor recruits you, they will likely give you RA right away (e.g., when you apply).  A common scenario is that you first get admitted with TA, and then after a year or two find an advisor to support you with RA. 


\begin{tcolorbox}[left=1pt,right=1pt,top=1pt,bottom=1pt]
Vu: If you're lucky and got recruited by a prof. who would give you RA right away, it's very likely you will get admitted.  For example, if a prof., even if not in PhD admission committee, wants to work with and funds you, the PhD admission committee will respect that decision and admit your application (unless your application has many red flags).
\end{tcolorbox}

\subsection{Fellowship/Scholarship}

Fellowship is another type of funding in which the student applies for (e.g., from school, industries, government). Fellowships are typically competitive and generous, and gives pretty much all benefits tuition/insurance that a TA/RA has.  Moreover, it often gives higher stipend (including summer) and opens doors for job opportunities (e.g., internship).  For example, a student with a Microsoft fellowship will likely get an internship at Microsoft.  

In general, fellowship is prestigious, and you will stand out if you get one.  Every PhD student has pubs, but only superstars have NSF grad or Microsoft fellowship. In fact, these are so prestigious that even if you didn't get it but make it to the final round, school will still mention you on their website and you still should put it on your CV.


\textbf{How to get Fellowship?} You apply for them.  The US government has many fellowships but these would likely require US citizenship or residency.  However, tech companies including Google, Microsoft, Facebook, IBM have fellowships that international students can apply for. 

Prestigious fellowships typically require a clear and good research plan, so it is a good idea to wait until at least your second year to have research experience and even publication before applying. Remember, you're competiting with the top Ph.D. students at top universities worldwide. 


\begin{tcolorbox}[left=1pt,right=1pt,top=1pt,bottom=1pt]
At GMU, Ph.D. applicants are automatically eligible for a Presidential Fellowship.  It is at least as good as GTA but the most important thing is that as a fellowship it is truly free money (i.e., you are not depending on any prof. or TA duties).  PhD admission committee members nominate applicants for this fellowship and the committee will vote and give the fellowship to the top 2.
\end{tcolorbox}


\section{Miscs and FAQs}

\begin{enumerate}
    \item   Is an MS degree required for admission to PhD?
    \begin{description}
    \item No. In fact, student with BS can get MS degree "along the way" to PhD.  However, MS can help if it gives research experience or is from a more well-known school than your undergrad institution.
    \end{description}

    \item Can I apply to PhD to CS if my undergrad was in Math, Biology, Finance, etc?
    \begin{description}
        \item Yes, as long as you can demonstrate you are ready for CS PhD research through research experiences, LoRs, statements, etc as mentioned.
    \end{description}
    
    \item What can you do to increase your admission chance?
    \begin{description}
        \item Show something that makes you \textbf{stand out}, e.g., are you a female or a minority in CS (research for this on Google)? Do you participate in outreach activities that help increase diversity and inclusion in CS?  All of these are important in CS in the US.
    
        Also, even if you do not have formal research experience, you can talk about your personal project. If it is used by many people, have lots of stars in Github, etc, it would certainly worth talking about.  If you write technical, research-like blogs, that would also help.
    \end{description}
    
    \item How do I address a professor Firstname Lastname?
    \begin{description}
        \item If you don't know that professor (e.g., first email contact), then use \textbf{Prof. Lastname} or \textbf{Dr. Lastname}. I've seen many international students write \textbf{Prof.} or \textbf{Dr.} \textbf{Firstname Lastname}.  Writing like that makes it like you copy and paste the names, so no need to do so,  just Prof. or Dr. Lastname.
        
        Also do not use Mr. or Mrs., or just write Firstname. May be it is OK with others but I find it a bit disrespectful. As you know that prof. better and depends on their preference, you can call them by their Firstname.


\begin{tcolorbox}[left=1pt,right=1pt,top=1pt,bottom=1pt]
    Vu: I've been called Dr. Vu and I find it a bit amusing but am totally fine with it.
    \end{tcolorbox}
    \end{description}
\end{enumerate}






\section{Acknowledgement}

The following people contributed to this document: Craig Yu (GMU), Hakan Aydin (GMU).





\bibliographystyle{abbrv}
\bibliography{phd-cs-us.bib}
\newpage

\end{document}




# - Asking professors if they can get in or if they take students.\\
#   > Many international students will write professors asking if they can
#   get in, or if they are taking students. Unlikely anyone will reply to
#   such email because the short answer is we simply cannot evaluate your
#   profile ourselves, it needs to go to the whole committee as explained.
#   Even if a professor is seeking for students (almost everyone is
#   looking for good students), they simply cannot take one in without
#   going through the application process.


# - See this link for a large list of [[https://github.com/dynaroars/dynaroars.github.io/wiki/Funding][funding sources for PhD students]]

# ** Self-funded
# - Some PhD students are funded through their company
#   - The students work full time for the company and do their PhD part-time
#   - Typically will take longer (because you have to work full time) and require some serious committement from the students

# - Some PhD students are funded through country, e.g., GMU has several students from Saudi Arabia that are funded through their government.

# ** Others  
# - There are also various kind of one-time thing and small awards from industry or organization, e.g., a $2000 award for the summer.
Unlike the B.A., where you or your parents pay many tens of thousands of dollars,
or the M.S., where you typically work as a teaching assistant and possibly continue
to pay many tens of thousands of dollars, the Ph.D. is a time where funding is not a
concern to you. At most schools, you will not pay tuition during the time that you
are getting a Ph.D.. Typically, you will also receive a living stipend – on the order
of $2000 per month, from which you will pay your living expenses. Ideally, your
only responsibility will be research. This is called doing an RAship (Research
Assistantship).
The Ph.D. is a tremendous opportunity. You get to pick an advisor in any re-
search area you like and then you get to do research in that area, receive mentoring,
think deeply on problems, publish papers, become famous, while paying zero tu-
ition for 6 years and receiving a salary. Your advisor is paying for this opportunity
by writing grant proposals to companies and to the government to ask for fund-
ing. A single graduate student can cost an advisor upwards of 80K per year (given
the cost of tuition, the stipend, the overhead tax charged by the school, cost of
equipment and physical space, etc.).
Important note 1: At most schools, you can only do an RAship if you have
an advisor who has funding for you. Since some advisors don’t apply for grants
or are in areas which aren’t well-funded, you may have to work as a teaching
assistant every semester to get your stipend. This is called a TAship (Teaching
Assistantship). When I was a graduate student, I had a few friends who were
forced to TA 13 semesters, to fund their way through school! Alternatively, you
will have to restrict your choice of advisors to those who have funding. At CMU,
every Ph.D. student is guaranteed a stipend plus tuition regardless of which advisor
she chooses to work with.
Important note 2: There are many companies and government organizations
which offer Graduate Fellowships for Ph.D. students. If you are lucky enough
to get one of these, they will cover most of your way through graduate school,
and you will never have to worry about whether your advisor has funding or not.
Details about graduate fellowships will be discussed in Section 4.

** Current 2023 Rankings of CS PhD programs in the U.S.
  Below is the 2023 Rankings from [[https://www.csrankings.org][CSRankings.org]] as of <2023-04-09 Sun>. 
  
1. Carnegie Mellon University 
2. Univ. of Illinois at Urbana-Champaign 
3. Univ. of California - San Diego  
4. Massachusetts Institute of Technology 
5. Stanford University 
5. University of Michigan  
7. University of Washington 
8. Cornell University  
8. Univ. of California - Berkeley  
10. Georgia Institute of Technology 
11. Northeastern University  
11. University of Maryland - College Park 
13. University of Wisconsin - Madison  
14. Purdue University  closed chart
15. University of Texas at Austin  
16. Columbia University  
16. University of Pennsylvania 
18. New York University  
18. Princeton University 
20. University of Massachusetts Amherst  
21. Univ. of California - Los Angeles
22. University of Chicago 
23. Stony Brook University
23. Univ. of California - Santa Barbara 
25. Rutgers University  
26. Univ. of California - Irvine 
27. University of Southern California 
28. Duke University 
29. Univ. of California - Riverside
30. Pennsylvania State University 
31. Northwestern University 
32. Ohio State University 
33. George Mason University 
33. Harvard University  
35. Brown University  
35. University of Utah 
37. Texas A&M University
37. Univ. of California - Santa Cruz  
37. Yale University  
40. Boston University 
40. University at Buffalo
40. University of Colorado Boulder 
43. North Carolina State University
44. Rice University 
44. University of Illinois at Chicago  
46. University of North Carolina 
46. University of Virginia 
48. Arizona State University 
48. University of Minnesota 
48. Virginia Tech  
51. Oregon State University  
51. Univ. of California - Davis
    
* Links
:PROPERTIES:
:CUSTOM_ID: links
:END:
- https://idleprocess.wordpress.com/2009/12/07/why-go-to-graduate-school-and-how-to-get-into-the-program-of-your-dreams/
- https://mycsphd.org/_pages/application-parts.html
- https://emeryberger.com/admission-notes/
- https://www.elsevier.com/connect/9-things-you-should-consider-before-embarking-on-a-phd
- https://grad.uchicago.edu/wp-content/uploads/2019/08/Applying-to-PhD-Programs-Guide-2019.pdf
- https://www.pathwaystoscience.org/pdf/CIC_GradSchoolGuide.pdf
- https://www.science.org/content/article/applying-phd-these-10-tips-can-help-you-succeed
- https://chrisblattman.com/blog/2022/03/25/faqs-on-phd-applications/
- https://www.cs.cmu.edu/~harchol/gradschooltalk.pdf



# * Types of Funding for CS PhDs 






# * How does GRA work?
# While you might get a 9-month $28K GRA stipend,  your advisor typically has to budget at least twice that much.  Because they have to pay your tuitition, insurance, and overhead that the university takes.  The chart below shows that each CS GRA costs about $70K at GMU.

# - Example budget of a PhD student in CS at GMU.  Using stipend and tuition rate in 2023. Conference Registration and Travel are estimates.
# - It is also **very likely** that GMU faculty won't have to pay student tuition starting Fall 2025, when this happens then we can take out  Tuition and total cost is 46.5K.  

# | Budget                                 | Cost    |                                                                                           |
# |----------------------------------------+---------+-------------------------------------------------------------------------------------------|
# | GRA (9-month)                          | 28K     |                                                                                           |
# | GRA (summer)                           | 9K      | 3-month, 20hrs/week                                                                       |
# | Fringe Benefits (7.3%)                 | 0       | for faculty mostly (non-wage expenses such as social security, medicare)                  |
# | **Total Salary**                       | 37K     |                                                                                           |
# | Health Insurance                       | 3K      | Full year                                                                                 |
# | Tuition (In-State) for Fall and Spring | 15K     | ($680/ Credit + $150/Student Fee/ Credit)* 9 credits = $7470 ($6120 + $1350) per semester |
# | **Total**                              | 18K     | Full year tuition + insurance                                                             |
# | Total Materials & Supplies             | 0       |                                                                                           |
# | Conference Registration                | 500     |                                                                                           |
# | International Travel                   | 1.8K    |                                                                                           |
# | Domestic Travel                        | 700     |                                                                                           |
# | **Total Travel**                       | 3K      |                                                                                           |
# | Total **Direct** Cost                  | 58K     | Salary 37K + Travel 3K  + Health 3K + Tuition 18K                                         |
# | F & A (MTDC)                           | 21K     | Direct Cost - GRA Salary                                                                  |
# | Total **Indirect** Cost                | 12K     | **58.9%** of MTDC                                                                         |
# | **Total** (Direct + Indirect)          | **70K** |                                                                                           |


# - F & A = Facilities & Administrative Cost Base 
# - MTDC = Modified Total Direct Cost 


# * Miscs

# - Asking professors if they can get in or if they take students.\\
#   > Many international students will write professors asking if they can
#   get in, or if they are taking students. Unlikely anyone will reply to
#   such email because the short answer is we simply cannot evaluate your
#   profile ourselves, it needs to go to the whole committee as explained.
#   Even if a professor is seeking for students (almost everyone is
#   looking for good students), they simply cannot take one in without
#   going through the application process.

# - How long for a CS PhD in US ? > 5-6 years on avg.

# - Does having an MS help? > It can help if the MS gives research
#   experience or is from a more well-known school than your
#   undergraduate. But not having an MS does not hurt either. Nowadays it
#   is normal for students with bachelor degrees apply directly to
#   PhD in CS.
